\input opmac

\typosize[12/20]

My initial project was to formalize L'H\^opital's theorem using only the definitions used in Assignment 5. However, after much debugging, I found that the definition of {\tt approaches\_at} provided in both Assignment 4 and 5 contains a bug. It defines $\lim_{x\to c}f(x)=L$ as $$\forall\epsilon>0,\exists \delta >0 : |x - c| < \delta \Longrightarrow |f(x)-L|<\epsilon.$$ This definition, however, is too strict can causes the subsequent definition of {\tt continuous} to be content free, since it does not allow $\lim_{x\to c}f(x)=L\wedge f(x)\ne L$---a fundamental property of limits. The correct definition is $$\forall\epsilon>0,\exists \delta >0 : 0<|x - c| < \delta \Longrightarrow |f(x)-L|<\epsilon,$$ which I have implemented. Debugging this fact was one of the harder parts of implementation, as well as implementing a utility lemma {\tt approaches\_at\_rw}, which allows users to rewrite the inside of limits using rewrite rules that need not apply at the limit point.

L'H\^opital's theorem is used to resolve ${0\over 0}$ indeterminate limits using derivatives when the derivative of the denominator is not 0. The proof of L'H\^opital's that I formalized was the classical one presented in undergraduate calculus classes:

\medskip
\noindent {\bf Definition}: I use the limit-difference-quotient definition of the derivative $$f'(c)=\lim{x\to c}{f(x)-f(c)\over x - c}$$ in this proof.

\medskip 
\noindent {\bf Theorem}: Given $f,g:{\bbchar R}\mapsto{\bbchar R}\in C^2$, $f(c)=g(c)=0$, and $g'(c)\ne0$, $$\lim_{x\to c}{f(x)\over g(x)}={f'(x)\over g'(x)}$$

\noindent{\it Proof\/}:
$$\eqalignno{
    \lim_{x\to c}{f(x)\over g(x)}&=\lim_{x\to c}{f(x)-0\over g(x)-0}\cr
    &=\lim_{x\to c}{f(x)-f(c)\over g(x)-g(c)}\cr
    &=\lim_{x\to c}{{f(x)-f(c)\over x-c}\over{g(x)-g(c)\over x-c}}\cr
    &=\lim_{x\to c}{\lim_{x\to c}{f(x)-f(c)\over x-c}\over\lim_{x\to c}{g(x)-g(c)\over x-c}}\cr
    &={f'(c)\over g'(c)}&\square
}$$

After proving L'H\^opital's theorem as given, I continued on to prove some dependencies such as the limit algebra laws. I have not completed the entire set needed since they are not the center of the midterm and each one is a complicated $\epsilon$-$\delta$ proof, but I have given proof outlines of the multiplication law and reproduced the proof of scalar summation that I gave in Assignment 4.

I intentionally did not use any of {\tt mathlib} other than {\tt data.real}, as I wanted to give completely elementary proofs from scratch of the theorems. This aspect of my proof also was quite challenging to deal with.

\bye